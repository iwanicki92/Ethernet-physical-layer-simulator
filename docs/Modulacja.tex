\section{Modulacja}

\subsection{Dlaczego potrzebujemy modulacji?}

Modulacja cyfrowa to technika zamiany bitów na sygnał oraz sygnału na bity. Jest kluczowym zagadnieniem w przesyle danych pomiędzy systemami komputerowymi. W odróżnieniu od modulacji analogowej, gdzie przesyłane dane wybierane są z przedziału,
modulacja cyfrowa operuje na dyskretnym zbiorze danych (bitach).

Dane reprezentowane są w postaci zmiany parametrów przesyłanego sygnału. Wyróżniane są cztery podstawowe metody:

\begin{enumerate}
    \item PSK (phase-shift keying) --- zmiana fazy fali nośnej sygnału
    \item FSK (frequency-shift keying) --- zmiana częstotliwości fali nośnej sygnału
    \item ASK (amplitude-shift keying) --- zmiana amplitudy fali nośnej sygnału
    \item QAM (quadrature amplitude modulation) --- połączenie PSK oraz FSK, a więc zmieniana jest zarówno amplituda oraz faza
\end{enumerate}

Zmiany sygnału (symbole) kodujące kolejne bity wybierane są ze skończonego zbioru nazywanego alfabetem modulacji.
Dział ten przedstawi popularne techniki modulacji w technologiach Ethernetowych

\subsection{Wprowadzenie}

Aby ławiej zrozumieć ideę stojącą za bardziej skomplikowanymi technikami modulacji, należałoby na wprowadzeniu wyjaśnić kilka podstawowych pojęć.

Główną charakterystyką łącza jest szerokość pasma (ang. bandwidth) --- określa ona maksymalną (teoretyczną) liczbę bitów jaką łączę jest w stanie przesłać w danym czasie. Podawana jest ona w bitach na sekundę [$bps$] lub w hercach [Hz].

Przepustowość (ang. channel capacity) --- rzeczywista szerokość pasma, zmierzona w określonych warunkach.

Przepływność (ang. bit rate) --- rzeczywista ilość bitów transmitowanych w jednostce czasu poprzez kanał, podawana również w $bps$ lub Hz. Jest stałą charakterystyką danego łącza.

W 1924 roku, Harry Nyquist przedstawił światu równanie, za pomocą którego określić można maksymalną przepływność łącza, które nie podlega szumom. 24 lata później, Claude
Shannon rozszerzył równanie Nyquista, uwzględniając szum. Udowodnił on, że maksymalną przepływność łącza, o szerokości pasma $B$ oraz stosunku sygnału do szumu $S/N$, można
obliczyć ze wzoru:

\begin{align*}
    \text{Przepływność}_{max} &= B * log_{2}{(1 + S/N)}
\end{align*}

Granica ta nazywana jest limitem Shannona.

Innym ważnym pojęciem jest multipleksacja (ang. multiplexing) i oznacza przesył wielu symboli jednocześnie w jednym kanale.

Na początku rozważmy przykład. Najprostszą metodą byłoby używanie dodatniego napięcia dla bitu równego 1 i ujemnego napięcia dla 0.
Technika ta nosi nazwę \textbf{NRZ (Non-Return-to-Zero)}. Nie jest ona wykorzystywana w praktyce --- nadając naprzemiennie 1 i 0 otrzymamy okres równy 2 bity, co oznacza że potrzebujemy pasma B/2 Hz przy prędkości B bit / s.
Nie trudno zauwarzyć, że do szybszego nadawnia, zwiększona musi zostać szerokość pasma, co nie jest optymalnym rozwiązaniem z uwagi na ograniczoność tego zasobu \cite{Computer-networks-Tanenbaum}.

Jednym z rozwiązań tego problemu jest wykorzystanie większej ilości poziomów napięcia. W powyższym przykładzie zastosowane zostały dwa poziomy, a co za tym idzie mamy do dyspozycji dwa symbole przesyłane przez kanał.
Zwiększenie poziomów do 4 dałoby nam 4 różne symbole, a więc 2 bity informacji. W rezultacie przepływność wzrosła dwukrotnie, natomiast szerokość pasma nie zmieniła się. Technika zadziała pod warunkiem, że strona odbiorcza dysponuje sprzętem, który pozwoli jej na
rozróżnienie wielu poziomów napięcia. Jednakże w praktyce jest to koszt, który jesteśmy w stanie ponieść.

