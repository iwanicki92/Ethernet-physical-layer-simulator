\section{Podsumowanie}

Wszystkie przyjęte cele pracy zostały osiągnięte, a więc:
\begin{itemize}
    \item stworzono symulator, który umożliwia następujące symulacje:
    \begin{itemize}
        \item symulację sygnału w skrętce z możliwością modyfikacji parametrów wejściowych oraz kanału,
        \item symulację modulacji NRZ, PAM4 oraz PAM16,
        \item symulację kodowania korekcyjnego Reeda-Solomona,
    \end{itemize}
    \item przygotowano instrukcję, wejściówkę oraz zadania laboratoryjne,
    \item przeprowadzono zajęcia dydaktyczne, wykorzystując podane powyżej materiały.
\end{itemize}

Dodatkowym sukcesem można określić intuicyjność symulatora, która była dla autorów istotnym założeniem podczas pracy nad rozwiązaniem. Uczestnicy zajęć nie zgłaszali żadnych problemów w obsłudze programu.

Przy planowaniu celów pracy, w porozumieniu z promotorem, zrezygnowano z pewnych elementów, jednak warto byłoby rozwinąć dyplom o:
\begin{itemize}
    \item symulację skramblowania i deskramblowania,
    \item symulację kodowania LDPC,
    \item symulację przesyłu sygnału światłowodem,
    \item narzędzie łączące rozwiązania Twisted-pair simulation oraz PAM16.
\end{itemize}