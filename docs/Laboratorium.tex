\section{Zajęcia dydaktyczne}
\subsection{Przebieg ćwiczenia}
\subsubsection{Wejściówka}

Na samym początku zajęć odbędzie się wejściówka składająca się z pytań otwartych
i zamkniętych.
Przykładowe pytania które mogą znaleźć się na wejściówce to:

\begin{enumerate}
    \item Podaj definicję dodawania i mnożenia w $\mathbb{F}_2$ bądź wypisz wynik tych
    działań dla wszystkich możliwych kombinacji elementów.
    \begin{itemize}
        \item Odpowiedź: Tablica~\ref{truth_table:title} bądź~(\ref{modulo_addition})
        oraz~(\ref{modulo_multiplication})
    \end{itemize}
    \item Ile błędnych symboli jest w stanie wykryć oraz poprawić kod Reeda-Solomona
    \begin{enumerate}[label=\Alph*:]
        \item wykryć: $n-k$, poprawić: $n-k-1$
        \item wykryć: $n-k-1$, poprawić: $n-k-1$
        \item wykryć: $\lfloor \frac{n-k}{2} \rfloor$, poprawić: $\lfloor \frac{n-k}{2} \rfloor$
        \item wykryć: $n-k$, poprawić: $\lfloor \frac{n-k}{2} \rfloor$
    \end{enumerate}
    \item Podaj zaletę oraz wadę stosowania większej ilości poziomów w modulacjach PAM
    \begin{itemize}
        \item Zaleta: jeden symbol koduje więcej bitów - więcej bitów jest przesyłanych w jednym cyklu zegara
        \item Wada: różnica między poszczególnymi symbolami maleje, a więc trudniej jest rozróżnić symbole
    \end{itemize}
    \item Czym się różni szerokość pasma od przepustowości?
    \item Opisz krótko czym jest NRZ (Non-Return-to-Zero)
    \item Czym jest przepływność łącza?
    \item Która biblioteka wykorzystywana jest do stworzenia GUI?
    \begin{enumerate}[label=\Alph*:]
        \item Tkinter
        \item PyQt --- Odpowiedź prawidłowa w punkcie~\ref{subsection:interfejs}
        \item OpenGL
        \item WindowsForms
    \end{enumerate}
\end{enumerate}
