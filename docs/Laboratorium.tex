\section{Zajęcia dydaktyczne}
\subsection{Wejściówka}

Na samym początku zajęć odbędzie się wejściówka składająca się z pytań otwartych
i zamkniętych.
Przykładowe pytania które mogą znaleźć się na wejściówce to:

\begin{enumerate}
    \item Podaj definicję dodawania i mnożenia w $\mathbb{F}_2$ bądź wypisz wynik tych
    działań dla wszystkich możliwych kombinacji elementów.
    \begin{itemize}
        \item Odpowiedź: Tablica~\ref{truth_table:title} bądź~(\ref{modulo_addition})
        oraz~(\ref{modulo_multiplication})
    \end{itemize}
    \item Czym się różni słowo kodowe wygenerowane kodem systematycznym i niesystematycznym
    \begin{itemize}
        \item słowo kodowe w kodowaniu systematycznym w przeciwieństwie do niesystematycznego zawiera w sobie kodowaną wiadomość~(\ref{subsection:Kod systematyczny})
    \end{itemize}
    \item Ile błędnych symboli jest w stanie wykryć lub poprawić kod Reeda-Solomona
    \begin{enumerate}[label=\Alph*:]
        \item wykryć: $n-k$, poprawić: $n-k-1$
        \item wykryć: $n-k-1$, poprawić: $n-k-1$
        \item wykryć: $\lfloor \frac{n-k}{2} \rfloor$, poprawić: $\lfloor \frac{n-k}{2} \rfloor$
        \item wykryć: $n-k$, poprawić: $\lfloor \frac{n-k}{2} \rfloor$
    \end{enumerate}
    \item Podaj zaletę oraz wadę stosowania większej ilości poziomów w modulacjach PAM
    \begin{itemize}
        \item Zaleta: jeden symbol koduje więcej bitów - więcej bitów jest przesyłanych w jednym cyklu zegara
        \item Wada: różnica między poszczególnymi symbolami maleje, a więc trudniej jest rozróżnić symbole
    \end{itemize}
    \item Czym się różni szerokość pasma od przepustowości?
    \item Opisz krótko czym jest NRZ (Non-Return-to-Zero)
    \item Czym jest przepływność łącza?
    \item Która biblioteka wykorzystywana jest do stworzenia GUI?
    \begin{enumerate}[label=\Alph*:]
        \item Tkinter
        \item PyQt --- Odpowiedź prawidłowa w punkcie~\ref{subsection:interfejs}
        \item OpenGL
        \item WindowsForms
    \end{enumerate}
\end{enumerate}

\subsection{Narzędzia}

\subsubsection{Symulator - wybór technologii}
Do stworzenia symulatora wybrano język programowania Python i wykorzystano między innymi następujące gotowe rozwiązania:

\begin{enumerate}
    \item PyQt będzie biblioteką wykorzystywaną do stworzenia interfejsu graficznego użytkownika (GUI) dla symulatora. PyQt zapewnia szeroki zakres narzędzi do tworzenia rozbudowanych i przyjaznych użytkownikowi interfejsów, co jest szczególnie ważne w symulatorze dydaktycznym, gdzie interfejs musi być intuicyjny i nie stanowić niepotrzebnego wyzwania lub problemu dla biorących udział w zajęciach,
    \item NumPy jest najpopularniejszą biblioteką Python implementującą algorytmy matematyczne. Między innymi oferuje generatory liczb pseudolosowych o różnych rozkładach, co jest wymagane do prawidłowego generowania ramek ethernetowych i błędów,
    \item  Matplotlib to popularna biblioteka do tworzenia wykresów, przydatna przy tworzeniu wykresów sygnałów,
    \item SPICE (Simulation Program with Integrated Circuit Emphasis) jest powszechnie stosowanym narzędziem do symulacji obwodów elektronicznych. Jest to rozbudowany program, który umożliwia modelowanie i analizę zachowania obwodów złożonych, takich jak układy analogowe, cyfrowe czy mikroelektroniczne. W celu korzystania z tego narzędzia w środowisku Python dostępna jest biblioteka PySpice, będąca interfejsem umożliwiającym korzystanie ze SPICE,
\end{enumerate}

\subsubsection{Interfejs użytkownika}
Interfejs użytkownika został wykonany przy użyciu PyQt5 oraz Qt Designer. Qt Designer to graficzne narzędzie do projektowania interfejsów użytkownika w ramach frameworka Qt. Umożliwia łatwe tworzenie i dostosowywanie wyglądu aplikacji oraz następne jego wygenerowanie jako kodu w języku Python lub C++.

Interfejs składa się z kilku zakładek stworzonych, które umożliwiają przełączanie między częściami aplikacji bez utraty wyników dotychczasowej pracy. Każda zakładka przeznaczona jest do innego zadania laboratoryjnego i zawiera symulacje innych rozwiązań ethernetowych. 

\begin{figure}[ht]
    \centering
    \includegraphics{images/zakladki.png}
    \caption{Zakładki symulatora}
    \label{fig:zakladki_image}
\end{figure}

Wykresy są tworzone przy pomocy biblioteki Matplotlib. Została dodatkowo stworzona klasa, która zawiera stworzone wykresy i może być użyta jako element graficznego interfejsu użytkownika, a więc dodana do niego.

\begin{figure}[ht]
    \centering
    \includegraphics[scale=0.5]{images/wykres.png}
    \caption{Wykres symulacji stworzony przy użyciu Matplotlib}
    \label{fig:wykres_image}
\end{figure}

\subsection{Ćwiczenie dydaktyczne --- modulacje PAM}

W trakcie zajęć laboratoryjnych student będzie miał do dyspozycji symulator phyether, który zawiera implementację wybranych rozwiązań
części standardów 25GBASE-T i 40GBASE-T.

\subsubsection{Wstęp teoretyczny}

Modulacja cyfrowa to technika zamiany bitów na sygnał oraz sygnału na bity. Jest kluczowym zagadnieniem w przesyle danych pomiędzy systemami
komputerowymi. Technologie Ethernetowe wykorzystują wiele technik modulacji. Ćwiczenie te ma na celu przybliżenie oraz porównanie
występujących technik modulacji PAM (Pulse-Amplitude Modulation), która jest jedną z najpopularniejszych w technologii Ethernet.

Najprostszym schematem modulacji jest NRZ --- Non-Return-to-Zero. Chcąc przesłać bit o wartości $0$, na skrętkę zostanie podany sygnał
ujemny, a w przypadku $1$ --- dodatni. Rozwiązanie te niesie za sobą parę wad. Przykładowo, gdy nadawane są długie ciągi zer lub jedynek, sygnał
nie ulega zmianie --- jest to zjawisko niepożądane podczas transmisji i może doprowadzić do desynchronizacji zegarów strony nadawczej i odbiorczej.
Z uwagi na to, NRZ wykorzystywany jest w praktyce w połączeniu z np. kodowaniem liniowym 64b/66b w celu uniknięcia sekwencji zer i jedynek.

PAM jest modulacją, w której dane przesyłane są w postaci zmian amplitudy sygnału. Zmiany te nazywane symbolami. Modulacje PAM różnią się między sobą
liczbą wykorzystywanych poziomów modulacji. PAM3 wykorzystuje trzy poziomy, PAM4 --- cztery, PAM16 --- szesnaście. NRZ można wobec tego nazwać PAM2.
Powodem dla którego zwiększenie poziomów ma sens jest zwiększona szybkość transmisji. Weźmy na przykład PAM4 --- mając cztery poziomy mamy
do dyspozycji cztery symbole $-3$, $-1$, $1$, $3$, a więc każdy symbol kodować może dwa bity danych. W przypadku NRZ, jeden symbol koduje tylko jeden bit.
Zatem zwiększenie liczby poziomów pozwala na przesył większej ilości bitów przy użyciu jednego symbolu. Schemat ten będzie działa o ile strona odbiorcza
potrafi rozróżnić poszczególne symbole od siebie, jest to jednak łatwiej osiągalne niż zwiększenie szerokości pasma.
Modulacje PAM4, PAM16 i inne, analogicznie jak NRZ, podatne są na długie ciągi zer i jedynek. Dlatego niezbęde jest zastosowanie różnych "gmatwaczy" bitów, które
zamienią takie sekwencje na bardziej zróżnicowany ciąg np. skrambler.

\subsubsection{Opis narzędzia}

Program phyether, w zakładce "PAM", posiada symulator modulacji NRZ, PAM4 oraz PAM16, który ilustruje zachowanie sygnału w skrętce podczas transmisji, w zależności
od wybranych modulacji.
Narzędzie będzie wykorzystywane podczas tej części ćwiczenia.

%\includegraphics[scale=0.37]{pam_tab.png}

Powyższa grafika prezentuje zakładkę "PAM" programu phyether. Nad przyciskiem \textbf{Simulate} znajduje się pole tekstowe, do którego wpisane mogą być dane w formie
liczb w systemie szesnastkowym. Po wpisaniu danych i kliknięciu \textbf{Simulate}, wpisane dane zamieniane są na symbole modulacji NRZ, PAM4 oraz PAM16, które
następnie wysyłane są na medium. Wynik symulacji przedstawiony jest w dolnej części zakładki.

\subsubsection{Zadania do realizacji}

\begin{enumerate}
    \item Zamień numer swojego indeksu na postać szesnastkową i wykorzystaj go jako liczbę do przesłania. Przeprowadź symulację. Zanotuj w sprawozdaniu
    przybliżony czas transmisji oraz liczbę poziomów natężenia. Opisz wnioski, które nasuwają Ci się po wykonanym ćwiczeniu.
    \item Prześlij ciąg składający się z samych jedynek (fffffff \dots). Popatrz na wynik symulacji. Jak nazywa się zaobserwowane zjawisko? Czy znasz sposoby,
    które zapobiegają jego wystąpieniu? Zanotuj w sprawozdaniu.
\end{enumerate}
