\section*{Streszczenie}
\addcontentsline{toc}{section}{\protect\numberline{}Streszczenie}

Celem pracy jest stworzenie symulatora wybranych rozwiązań warstwy fizycznej sieci Ethernet oraz przeprowadzenie zajęc laboratoryjnych z grupą studentów. Wybrane rozwiązania to: kodowanie korekcyjne Reeda-Solomona oraz modulacja amplitudy impulsów (PAM). Dodatkowo zakres prac ograniczono do skrętki oraz konkretnego standardu - 40GBASE-T. Pierwszy rozdział wprowadza do kodowania korekcyjnego, przedstawiając wymaganą teorię matematyczną, jego właściwości oraz zastosowania w standardach Ethernet. Następnie przedstawiono potrzebę wykorzystania modulacji i różne jego rodzaje. Dla każdej modulacji wymienione są zalety i wady oraz wykorzystanie w standardach Ethernet. W kolejnym rozdziale obszernie opisany jest standard 40GBASE-T. Przybliża między innymi: możliwości tego standardu oraz techniki i rozwiązania, które są w nim wykorzystane, . Następna część dotyczy stworzonego symulatora. Zawiera przegląd dostępnych narzędzi oraz tych, które rzeczywiście wybrano, w tym: język programowania i biblioteki, Ostatnią częścią jest sprawozdanie z prowadzonego laboratorium. Opisuje zarówno przebieg samych zajęć dydaktycznych, jak i przygotowanie sali i sprzętu. Do pracy zostały załączone dodatki: instrukcja laboratoryjna, kod symulatora oraz instrukcja instalacji wraz z wymaganymi plikami na dołączonym urządzeniu przenośnym.

Słowa kluczowe: Ethernet, symulacja, laboratorium, skrętka, kodowanie korekcyjne Reeda-Solomona, PAM, Python.