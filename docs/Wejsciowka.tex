\setcounter{secnumdepth}{0}
\section*{Dodatek B: Wejściówka}
\addcontentsline{toc}{section}{\protect\numberline{}Dodatek B: Wejściówka}

\begin{enumerate}
    \item Ile błędnych symboli jest w stanie wykryć lub poprawić kod Reeda-Solomona?
    \begin{enumerate}[label=\Alph*)]
        \item wykryć: $n-k$, poprawić: $n-k-1$
        \item wykryć: $n-k-1$, poprawić: $n-k-1$
        \item wykryć: $\lfloor \frac{n-k}{2} \rfloor$, poprawić: $\lfloor \frac{n-k}{2} \rfloor$
        \item wykryć: $n-k$, poprawić: $\lfloor \frac{n-k}{2} \rfloor$
    \end{enumerate}
    \item Podaj zaletę oraz wadę stosowania większej ilości poziomów w modulacjach PAM.\\ \\ \\ \\ \\
    \item Opisz krótko czym jest NRZ (Non-Return-to-Zero).\\ \\ \\ \\
    \item Podaj definicję dodawania i mnożenia w $\mathbb{F}_2$ bądź wypisz wynik tych działań dla wszystkich możliwych kombinacji elementów. \\ \\ \\ \\
    \item Czym się różni słowo kodowe wygenerowane kodem systematycznym i niesystematycznym?
\end{enumerate}
