\setcounter{secnumdepth}{0}
\section*{Dodatek C: Zadania laboratoryjne}
\addcontentsline{toc}{section}{\protect\numberline{}Dodatek C: Zadania laboratoryjne}

\begin{enumerate}
    \item Zakładka: Reed-Solomon, ustawienia programu: kod systematyczny BCH, $n=15$, $GF=2^4$.
    Dla $k \in \{ 2, 6, 10, 13 \}$ sprawdź wartość słowa kodowego dla k-symbolowej wiadomości zawierającej same zera. Dlaczego otrzymałeś takie słowa kodowe?\\ \\ \\ \\ \\
    \item Zakładka: Reed-Solomon, ustawienia programu: $n = 7$, $k = 3$, $GF = 2^3$,
    kod systematyczny BCH, tryb dziesiętny, wiadomość wejściowa `1 2 3'.
    Sprawdź dla błędów `3', `3 2', `3 2 1' oraz `3 2 1 4' czy dekoder jest w stanie
    wykryć błędy oraz czy jest w stanie je poprawić a jeżeli tak to czy poprawnie.
    Czy wiesz dlaczego pojawiły się rozbieżności między błędami poprawionymi a wykrytymi? \\ \\ \\
    \begin{table}[h]
        \renewcommand{\arraystretch}{1.8}
        \centering
        \begin{tabular}{|c|c|c|>{\centering\arraybackslash}p{5cm}|}
            \hline
            \textbf{Błąd} & \textbf{Wykrywa [TAK/NIE]} & \textbf{Poprawia [Ile]} & \textbf{Poprawia poprawnie [TAK/NIE]} \\
            \hline
            3 & & & \\
            \hline
            3 2 & & & \\
            \hline
            3 2 1 & & & \\
            \hline
            3 2 1 4 & & & \\
            \hline
        \end{tabular}\label{tab:rs2}
    \end{table}
    \item Zakładka: Reed-Solomon Shift Register, ustawienia programu:
    $n = 7$, $k = 3$, $GF = 2^3$. Oblicz wielomian prymitywny i generator naciskając
    przycisk `Calculate primitive poly/element'.
    Zakoduj wiadomość: `1 2 3 4 5' w symulatorze po czym zakoduj wiadomość
    używając wzoru z sekcji `Systematyczny kod BCH'. Porównaj wyniki.
    \newpage
    \item Zakładka: Reed-Solomon, ustawienia programu: $n = 15$, $k=7$, $GF = 2^4$,
    kod systematyczny BCH.
    Zakoduj dowolną niezerową $k$-symbolową wiadomość. Sprawdź czy przesunięcia słowa kodowego
    (z użyciem strzałek przy słowie `encoded') także będą słowem kodowym. \\ \\ \\ \\ \\
    \item Zamień numer swojego indeksu na postać szesnastkową i wykorzystaj go jako liczbę do przesłania. Przeprowadź symulację. Zanotuj w sprawozdaniu
    przybliżony czas transmisji oraz liczbę poziomów natężenia. Opisz wnioski, które nasuwają Ci się po wykonanym ćwiczeniu.  \\ \\ \\ \\ \\
    \item Prześlij ciąg składający się z samych jedynek (fffffff \dots). Popatrz na wynik symulacji. Jak nazywa się zaobserwowane zjawisko? Czy znasz sposoby,
    które zapobiegają jego wystąpieniu? Zanotuj w sprawozdaniu.
\end{enumerate}
