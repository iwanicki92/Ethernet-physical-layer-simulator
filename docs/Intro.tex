\section{Wstęp}
\subsection{Wprowadzenie}

Celem projektu jest opracowanie programu umożliwiającego przeprowadzenie symulacji wybranych rozwiązań warstwy fizycznej Ethernet.
Przyjęty, wraz z promotorem, zakres prac zakłada realizację następujących elementów:
\begin{itemize}
    \item symulację sygnału w skrętce z możliwością modyfikacji parametrów wejściowych oraz kanału,
    \item symulację wybranch modulacji PAM,
    \item symulację kodowania korekcyjnego Reeda-Solomona.
\end{itemize}

Ponadto w ramach pracy inżynierskiej dyplomanci przygotują scenariusz zajęć laboratoryjnych wykorzystujących opracowany symulator, jak
również przeprowadzą te zajęcia z grupą studentów. Scenariusz zawrze takie elementy, jak wstęp teoretyczny do omawianych zagadnień,
zadania na wejściówkę, zadania do realizacji na laboratoriach oraz opracowanie wymienionych zadań.

Podczas przygotowywania dyplomu autorzy zapoznają się ze standardem IEEE 802.3, specyfiką oraz problemami występującymi w warstwie fizycznej
w sieciach Ethernet. Dzięki realizacji tych zadań dyplomanci poszerzą swoją wiedzę w wymienionych obszarach oraz podzielą się wynikami
swojej pracy ze studentami.

\subsection{Autorstwo fragmentów pracy}

\begin{itemize}
    \item Michał Iwanicki
    \begin{itemize}
        \item Kodowanie Reeda-Solomona --- rozdział 4, wstęp teoretyczny dla studentów
        wraz z pytaniami na wejściówkę i zadaniami, implementacja w kodzie,
        implementacja interfejsu graficznego
        \item Symulacja sygnału --- implementacja symulacji w kodzie, częściowe
        autorstwo interfejsu.
    \end{itemize}
    \item Marcin Garnowski
    \begin{itemize}
        \item Symulator --- rozdział 7,
        \item Zajęcia dydaktyczne --- sprawozdanie --- rozdział 8, częściowe autorstwo,
        \item Opis symulatora w instrukcji,
        \item Pytanie na wejściówkę dotyczące symulatora,
        \item Dodatki C, D, E,
        \item Częściowe autorstwo interfejsu. \\
    \end{itemize}
    \item Mateusz Bauer
    \begin{itemize}
        \item Zajęcia dydaktyczne --- sprawozdanie --- rozdział 8, częściowe autorstwo
        \item 25GBASE-T i 40GBASE-T, rozdział 6
        \item Modulacja, rozdział 5, wstęp teoretyczny do zajęć, pytania na wejściówkę,
        zadania laboratoryjne, implementacja interfejsu graficznego oraz implementacja
        modulacji w kodzie
    \end{itemize}
\end{itemize}
