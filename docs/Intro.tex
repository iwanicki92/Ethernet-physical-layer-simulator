\section{Wstęp}
\subsection{Wprowadzenie}

Celem projektu jest opracowanie programu umożliwiającego przeprowadzenie symulacji wybranych rozwiązań warstwy fizycznej Ethernet.
Przyjęty, wraz z promotorem, zakres prac zakłada realizację następujących elementów:
\begin{itemize}
    \item symulację sygnału w skrętce z możliwością modyfikacji parametrów wejściowych oraz kanału,
    \item symulację wybranch modulacji PAM,
    \item symulację kodowania korekcyjnego Reeda-Solomona.
\end{itemize}

Ponadto w ramach pracy inżynierskiej dyplomanci przygotują scenariusz zajęć laboratoryjnych wykorzystujących opracowany symulator, jak
również przeprowadzą te zajęcia z grupą studentów. Scenariusz zawrze takie elementy, jak wstęp teoretyczny do omawianych zagadnień,
zadania na wejściówkę, zadania do realizacji na laboratoriach oraz opracowanie wymienionych zadań.

Podczas przygotowywania dyplomu autorzy zapoznają się ze standardem IEEE 802.3, specyfiką oraz problemami występującymi w warstwie fizycznej
w sieciach Ethernet. Dzięki realizacji tych zadań dyplomanci poszerzą swoją wiedzę w wymienionych obszarach oraz podzielą się wynikami
swojej pracy ze studentami.
