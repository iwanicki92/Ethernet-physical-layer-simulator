\section{Środowisko programistyczne}
\subsection{Język programowania}

Do stworzenia symulatora wybrano język programowania Python z uwagi na kilka istotnych powodów. Przede wszystkim, czytelność składni stanowi ogromne ułatwienie podczas wspólnego tworzenia oprogramowania, a prostota pozwala skupić się na istocie problemu, nie tracąc czasu na pokonywanie trudności języka.

Dodatkowo, wybór Pythona jest motywowany chęcią rozwijania naszych umiejętności w tym środowisku, zarówno na poziomie indywidualnym, jak i zawodowym. Python cieszy się dużą popularnością jako uniwersalny język programowania, używany w różnych dziedzinach, takich jak analiza danych, sztuczna inteligencja czy aplikacje webowe. Posiadanie umiejętności programowania w Pythonie otwiera drzwi do szerszych możliwości zawodowych i dostępu do różnorodnych ciekawych projektów.

Jednym z najważniejszych argumentów przemawiających za wyborem Pythona jest jego ogromna popularność. Związana z tym społeczność programistyczna tworzy rozbudowany ekosystem, oferujący dostęp do wielu gotowych rozwiązań, bibliotek i frameworków. W kontekście tworzenia symulatora, istnieje wiele bibliotek w Pythonie, które mogą okazać się niezwykle przydatne. Na przykład, biblioteki umożliwiające tworzenie interfejsów graficznych ułatwią korzystanie z symulatora, biblioteki do analizy i przetwarzania sygnałów pomogą modelować różne aspekty transmisji, a biblioteki do wizualizacji pozwolą na przedstawienie wyników w przystępny sposób.

Inną cechą, która wyróżnia ten język programowania, jest jego przenośność. To ma dla nas duże znaczenie przy tworzeniu symulatora, który musi działać w warunkach laboratoryjnych, a więc na dowolnym popularniejszym systemie operacyjnym oraz charakteryzować się łatwością instalacji. Te wymagania Python w naszej ocenie spełnia.

\subsection{Narzędzia, biblioteki i moduły}
Jednym z celów postawionych przez promotora jest wykorzystanie gotowych rozwiązań podczas pracy nad symulatorem. W tym rozdziale zostaną przedstawione biblioteki i moduły języka Python oraz inne narzędzia, które mogą zostać wykorzystane w programie.

Python oferuje wiele bibliotek, które mogą okazać się kluczowe: od interfejsu graficznego po gotowe narzędzia do symulacji. Oto przegląd kilku z nich, na które się zdecydowano:

\begin{enumerate}
    \item PyQt będzie biblioteką wykorzystywaną do stworzenia interfejsu graficznego użytkownika (GUI) dla symulatora. PyQt zapewnia szeroki zakres narzędzi do tworzenia rozbudowanych i przyjaznych użytkownikowi interfejsów, co jest szczególnie ważne w symulatorze dydaktycznym, gdzie interfejs musi być intuicyjny i nie stanowić niepotrzebnego wyzwania lub problemu dla biorących udział studentów
    \item NumPy jest najpopularniejszą biblioteką Python implementującą algorytmy matematyczne. Między innymi oferuje generatory liczb pseudolosowych o różnych rozkładach, co jest wymagane do prawidłowego generowania ramek ethernetowych i błędów
    \item  Matplotlib to popularna biblioteka do tworzenia wykresów. Może okazać się przydatna przy tworzeniu wykresów sygnałów
\end{enumerate}


Istnieją również inne popularne narzędzia, które mogą być użyteczne do symulacji rozwiązań warstwy fizycznej sieci Ethernet:
\begin{enumerate}
    \item SPICE (Simulation Program with Integrated Circuit Emphasis) jest powszechnie stosowanym narzędziem do symulacji obwodów elektronicznych. Jest to rozbudowany program, który umożliwia modelowanie i analizę zachowania obwodów złożonych, takich jak układy analogowe, cyfrowe czy mikroelektroniczne. W celu korzystania z tego narzędzia w środowisku Python dostępna jest biblioteka PySpice, będąca interfejsem umożliwiającym korzystanie ze SPICE,
    \item MATLAB to znane i powszechnie używane narzędzie do obliczeń numerycznych, analizy danych i modelowania systemów. Posiada szeroki zakres narzędzi i funkcji przeznaczonych do tworzenia modeli matematycznych, symulacji dynamicznych itp.,
    \item Scapy umożliwia tworzenie i przetwarzanie różnego rodzaju pakietów sieciowych, w tym ramek Ethernet, co jest kluczową funkcjonalnością symulatora.
\end{enumerate}

\section{Symulator}
\subsection{Interfejs użytkownika}\label{subsection:interfejs}
Interfejs użytkownika został wykonany przy użyciu PyQt5 oraz Qt Designer. Qt Designer to graficzne narzędzie do projektowania interfejsów użytkownika w ramach frameworka Qt. Umożliwia łatwe tworzenie i dostosowywanie wyglądu aplikacji oraz następne jego wygenerowanie jako kodu w języku Python lub C++.

Interfejs składa się z kilku zakładek stworzonych, które umożliwiają przełączanie między częściami aplikacji bez utraty wyników dotychczasowej pracy. Każda zakładka przeznaczona jest do innego zadania laboratoryjnego i zawiera symulacje innych rozwiązań ethernetowych.

Wykresy są tworzone przy pomocy biblioteki Matplotlib. Została dodatkowo stworzona klasa, która zawiera stworzone wykresy i może być użyta jako element graficznego interfejsu użytkownika, a więc dodana do niego.

\subsection{Symulacje wybranych rozwiązań}
Aplikacja umożliwa symulowanie wybranych rozwiązań fizycznej warstwy sieci Ethernet. W każdym przypadku użytkownik ma swobodę podawania własnych parametrów wejściowych, zmieniania ich, co ma na celu ułatwienie zrozumienia działania tych rozwiązań.

\subsubsection{Kodowanie korekcyjne Reeda-Solomona}
Kodowanie korekcyjne Reed-Solomona to metoda kodowania korekcyjnego, mająca na celu wykrywanie i naprawianie błędów w przesyłanych danych. Stworzona została w 1960 roku przez dwóch amerykańskich matematyków: Irving S. Reed i Gustave Solomon. Od tego czasu znalazła szerokie zastosowanie w dziedzinie komunikacji, kodowaniu i obsłudze dysków.

Główną ideą kodowania korekcyjnego Reed-Solomona jest dodawanie nadmiarowych danych do przesyłanych informacji, dzięki czemu w przypadku wystąpienia błędów, możliwe jest ich wykrycie i skorygowanie. Algorytm opiera się na algebraicznych właściwościach ciał skończonych, co umożliwia efektywne wykonywanie operacji matematycznych potrzebnych do kodowania i dekodowania.

W symulatorze kodowanie i dekodowanie wykorzytuje metody klasy ReedSolomon udostępnionej w bibliotece galois. Jest ona rozszerzeniem, dodającym operacje na ciałach skończonych, innej popularnej biblioteki języka Python - NumPy. Jej nazwa pochodzi od nazwiska francuskiego matematyka Évariste Galois, który zasłynął badaniami ciał skończonych, które nazywane są również ciałami Galois.

\subsubsection{Symulacja przesyłu sygnału}
Symulator umożliwia symulację przesyłu danych przez skrętkę lub kablem Ethernet, w której można śledzić zmiany napięć. W tym przypadku wykorzystana została biblioteka PySpice. Dzięki niej można nadać wykorzystywanemu przewodowi pożądane parametry, między innymi: opór, długość i indukcyjność.

Użytkownik ma możliwość podawania tych parametrów w przewijalnym oknie, gdzie może dodawać również kolejne przewodniki.

Symulacja wykonywana jest w osobnym wątku. Dodatkowo możliwa jest jednoczesna symulacja wielu przewodów o różnych parametrach, które następnie przedstawiane są na jednym wykresie. Opcjonalnie można ukrywać lub pokazywać wybrane symulacji zaznaczając odpowiednie pola przy listach parametrów odpowiednich przewodów.
