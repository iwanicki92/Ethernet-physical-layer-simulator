\setcounter{secnumdepth}{0}
\section*{Dodatek G: Symulacja tłumienia}
\addcontentsline{toc}{section}{\protect\numberline{}Dodatek G: Symulacja tłumienia}

\subsection{Wstęp}
Symulacja przejścia sygnału przez kanał została zrealizowana za pomocą symulatora
SPICE jednak użyty model linii transmisyjnej nie symulował tłumienia.
Próba zaimplementowania własnego modelu skończyła się niepowodzeniem z powodu
dużych częstotliwości co powodowało bardzo długie działanie programu ngspice i
ewentualne nagłe zakończenie działania programu z powodu błędów.

\subsection{Obliczanie tłumienia}
Tłumienie zostało obliczone za pomocą wzoru~(\ref{tlumienie})~\cite[sekcja 4.3.4.7]{TIA/EIA-568-B}
\begin{align}
    k_1 \cdot \sqrt{f} + k_2 \cdot f + \frac{k_3}{\sqrt{f}}\label{tlumienie}
\end{align}
gdzie $f$ to częstotliwość, $k_1$ to straty spowodowane efektem naskórkowości,
$k_2$ to straty spowodowane materiałem przewodnika a $k_3$ to straty spowodowane
przez użycie ekranowania

Wartości $k_1$, $k_2$, $k_3$ dla różnych kabli podane są w tabeli~\ref{tab:tlumienie}
\begin{table}
    \captionof{table}{Stałe tłumienia}
    \centering
    \begin{tabular}{c c c c}
        \toprule
        \textbf{Kabel} & \textbf{$k_1$} & \textbf{$k_2$} & \textbf{$k_3$} \\
        \toprule
        Cat5~\cite{IEEE-cabling} & 1.9108 & 0.0222 & 0.2 \\
        \midrule
        Cat5e~\cite[sekcja 4.3.4.7]{TIA/EIA-568-B} & 1.967 & 0.023 & 0.05 \\
        \midrule
        Cat6~\cite{IEEE-cabling} & 1.82 & 0.0169 & 0.25 \\
        \midrule
        Cat7~\cite{IEEE-cabling} & 1.8 & 0.01 & 0.2
    \end{tabular}\label{tab:tlumienie}
\end{table}
