\newcommand{\modulo}[1]{\mathbb{Z} / #1 \mathbb{Z}}
\section{Kodowanie korekcyjne Reeda-Solomona}
\subsection{Wstęp}

Kodowanie korekcyjne Reeda-Solomona zostało stworzone przez Irvina S. Reeda
oraz Gustava Solomona w 1960 roku~\cite{Reed-Solomon-original}
Kody Reeda-Solomona charakteryzują się 3 parametrami, rozmiarem alfabetu $q$
interpretowanym w ciele skończonym $\mathbb{F}_q$, długością wiadomości do
zakodowania $k$ oraz długością słowa kodowego $n$ gdzie $k < n \leq q$ oraz
$q=p^m$ gdzie $p$ to liczba pierwsza a $m \in \{ 2, 3, \ldots \}$

\subsection{Ciało skończone $\mathbb{F}_{q}$}

Aby zrozumieć działanie kodu Reeda-Solomona trzeba najpierw zrozumieć czym jest
ciało skończone $\mathbb{F}_q$ zwane też ciałem Galois $\operatorname{GF}(q)$.
Ciało to jest ciałem $K$ rzędu $q$ czyli takie które zawiera jedynie $q$ elementów.
Aby struktura algebraiczna była ciałem musi definiować 2 operacje zwane
dodawaniem i mnożeniem.
Te operacje muszą spełniać kilka warunków:

{\small
    \begin{align}
        a + (b + c) &= (a + b) + c & \forall a,b,c &\in K && \text{Łączność dodawania} \\
        a*(b*c) &= (a*b)*c & \forall a,b,c &\in K && \text{Łączność mnożenia} \\
        a + b &= b + a & \forall a,b &\in K && \text{Przemienność dodawania} \\
        a * b &= b * a & \forall a,b &\in K && \text{Przemienność mnożenia} \\
        a + 0 &= a & \forall a &\in K && \text{Element neutralny (0) dodawania} \\
        a * 1 &= a & \forall a &\in K && \text{Element neutralny (1) mnożenia} \\
        a + (-a) &= 0 & \forall a &\in K && \text{Element odwrotny (-a) dodawania} \\
        a*a^{-1} &= 1 & \forall a &\in K \setminus \{ 0 \} &&
            \text{Element odwrotny (} a^{-1} \text{) mnożenia} \label{field_mul_inverse}\\
        a*(b + c) &= (a*b) + (a*c) & \forall a,b,c &\in K &&
            \text{Rozdzielność mnożenia względem dodawania}
    \end{align}
}%
W kodzie solomona wykorzystujemy ciało $\mathbb{F}_2$ oraz ciało rozszerzone
$\mathbb{F}_{2^m}, m \in \{ 1, 2, 3, \ldots \}$.
Aby stworzyć ciało rzędu $p$ gdzie $p$ jest liczbą pierwszą można wykorzystać
pierścień klas reszt $\mathbb{Z} / p \mathbb{Z}$ z elementami~(\ref{modulo_elements})
i działaniem dodawania~(\ref{modulo_addition}) i mnożenia~(\ref{modulo_multiplication})

\begin{align}
    \modulo{p} = \{ [a]_p | a \in \mathbb{Z} \} = \{ [0]_p, [1]_p,
    [2]_p, \ldots, [p-1]_p \} \label{modulo_elements}
\end{align}
\begin{align}
    [a]_p + [b]_p &= [a + b]_p \label{modulo_addition} \\
    [a]_p * [b]_p &= [a * b]_p \label{modulo_multiplication}
\end{align}

Dla $p$ niebędących liczbami pierwszymi $\modulo{p}$ nie będzie
ciałem skończonym, ponieważ nie wszystkie elementy będą spełniały
warunek~(\ref{field_mul_inverse}).
Dla $\mathbb{F}_2$ działania $+$ i $*$ są równoważne operacjom logicznym XOR
oraz AND zdefiniowanymi w tablicy~\ref{truth_table:title}
\begin{table}
    \captionof{table}{Dodawanie i mnożenie w $\mathbb{F}_2$}\label{truth_table:title}
    \centering
    \begin{tabular}{c c | c c}
        \toprule
        a & b & + & * \\
        \midrule
        0 & 0 & 0 & 0 \\
        \midrule
        0 & 1 & 1 & 0 \\
        \midrule
        1 & 0 & 1 & 0 \\
        \midrule
        1 & 1 & 0 & 1 \\
        \bottomrule
    \end{tabular}
\end{table}

\subsection{Rozszerzone ciało skończone $\mathbb{F}_{2^m}$}

Aby stworzyć ciało skończone o rzędzie $2^m$, $k \in \{ 1, 2, 3, \ldots \}$ musimy
najpierw znaleźć nierozkładalny wielomian $p(x)$ stopnia $k$ o współczynnikach
$c_{n} \in \mathbb{F}_{2}$.
Elementami tego ciała będą wielomiany o postaci $c_{0} + c_{1}\alpha + c_{2}\alpha^{2} +
    \cdots + c_{k-1}\alpha^{k-1}$, $c_{n} \in \{0, 1\}$.
Zbiór tych elementów można zapisać jako zbiór wielomianów~(\ref{gf_extended:polynomial}),
zbiór k-krotek lub wartości binarnych zawierających współczynniki
wielomianu~(\ref{gf_extended:binary}).
\begin{align}
    \left\{ \sum_{n=0}^{k-1} c_{n}\alpha^{n} \,|\, c_{n} \in \{0,1\} \text{ for } 1 \le n \le k \right\}
        = \{ 0, 1, \alpha, 1 + \alpha, \alpha^{2}, \ldots, 1 + \alpha + \alpha^2 + \cdots + \alpha^{k-1} \}
        \label{gf_extended:polynomial} \\
    \{ 0, 1 \}^{k} = \{ (0, 0, 0, \ldots, 0), (1, 0, 0, \ldots, 0), (0, 1, 0, \ldots, 0),
    (1, 1, 0, \ldots, 0), \ldots, (1, 1, 1, \ldots, 1) \} \label{gf_extended:binary}
\end{align}

\subsection{Wykorzystanie w standardach Ethernetowych}

Różne kody Reeda-Solomona są wykorzystywane w wielu standardach Ethernet,
wyróżnione w tablicy~\ref{standards:title}

\begingroup
\hyphenpenalty10000
\exhyphenpenalty10000
\begin{table}[h]
\captionof{table}{Kodowania RS w różnych standardach~\cite{Ethernet}}\label{standards:title}
\centering
    \begin{tabular}{m{3cm} m{9cm}}
    \toprule
    Kodowanie RS    & Standardy                                                         \\
    \midrule
    RS(528,514)     & 10GBASE-R, 25GBASE-R, 100GBASE-CR4, 100GBASE-KR4, 100GBASE-SR4    \\
    \midrule
    RS(544,514)     & 50GBASE-R, 100GBASE-KP4, 100GBASE-CR2, 100GBASE-SR2, 100GBASE-DR, 100GBASE-FR1, 100GBASE-LR1, 200GBASE-R, 400GBASE-R \\
    \midrule
    RS(450,406)     & 1000BASE-T1                                                       \\
    \midrule
    RS(192,186)     & 25GBASE-T, 40GBASE-T                                              \\
    \midrule
    RS(360,326)     & 2.5GBASE-T1, 5GBASE-T1, 10GBASE-T1                                \\
    \bottomrule
    \end{tabular}
\end{table}
\endgroup

\subsection{Właściwości kodu}
Kody Reeda-Solomona cechują się możliwością korekty $\lfloor \frac{n-k}{2} \rfloor$
oraz wykrycia $n-k$ błędnych symboli. Symbol w ciele $\mathbb{F}_{2^m}$ składa się
z $m$ bitów co w przypadku błędów grupowych daje możliwość korekty maksymalnie
$m \cdot \lfloor \frac{n-k}{2} \rfloor$ bitów bądź detekcji $p(n-k)$ przekłamanych
bitów

\subsection{Tworzenie kodu}
Istnieje wiele różnych sposobów tworzenia kodu które tworzą kod o innych właściwościach.


\subsubsection{Oryginalny sposób}
Sposób kodowania przedstawiony w pracy Reeda i Solomona polega na stworzeniu wielomianu $p_m(x)=\sum_{i=0}^{k-1}m_{i}x^i$, gdzie $m_i\in\mathbb{F}_q$ to $i$\nobreakdash-ty element wiadomości, po czym za pomocą tego wielomianu obliczane jest słowo kodowe $C(m)=(p_m(a_0), p_m(a_1), \ldots, p_m(a_{n-1}))$ gdzie $a_i$ to różne elementy ciała $\mathbb{F}_q$.

\subsubsection{Kod systematyczny}
Za pomocą niewielkiej modyfikacji można stworzyć kod systematyczny czyli taki w którym słowo kodowe zawiera w sobie kodowaną wiadomość.
Żeby stworzyć kod systematyczny musimy zmodyfikować sposób tworzenia wielomianu w taki sposób by $p_m(x_i)=m_i$ dla $i \in \{0,1,\ldots,k-1\}$.

Jednym ze sposobów stworzenia takiego wielomianu jest użycie metody interpolacji wielomianów. Słowo kodowe wygenerowane z tego wielomianu będzie zawierało wiadomość w pierwszych $k$ elementach.
\[C(m)=(p_m(a_0), p_m(a_1), \ldots, p_m(a_{n-1}))=(m_0, m_1, \ldots, m_{k-1}, p_m(a_k), p_m(a_{k+1}), \ldots, p_m(a_{n-1}))\]

\subsection{Dekodowanie}
\subsubsection{Algorytm Berlekampa-Welcha}
W roku 1986 Lloyd R. Welch oraz Elwyn R. Berlekamp uzyskali patent na dekoder umożliwiający uzyskanie oryginalnego wielomianu $p_m(x)$ oraz wielomianu $E(x)$ który zwraca 0 dla punktów $x$ w których nastąpiło przekłamanie~\cite{Berlekamp-Welch}
