\section*{Abstract}
\addcontentsline{toc}{section}{\protect\numberline{}Abstract}

The purpose of the work is to create a simulator of selected solutions of the physical layer of the Ethernet network and conduct laboratory classes with a group of students. The selected solutions are: Reed-Solomon error-correcting code and pulse-amplitude modulation (PAM). In addition, the scope of work was limited to twisted-pair and specific standard - 40GBASE-T. The first chapter introduces error-correcting coding, providing the required mathematical theory, its properties and applications in Ethernet standards. Afterwards, the need to use modulation and its different types are presented. For each modulation, the advantages, disadvantages and use in Ethernet standards are listed. The next chapter extensively describes the 40GBASE-T standard. It introduces, among other things: the capabilities of this standard and the techniques as well as solutions that are used in it. The next section deals with the simulator. It provides an overview of the available tools and those that were actually chosen, including: programming language and libraries, The last part is a report on the conducted laboratory classes. It describes both the course of the study itself and the preparation of the class room and equipment. The following appendices have been attached to the work: a laboratory instruction, simulator code and installation guide with the required files on the included data storage device.

Keywords: Ethernet, simulation, laboratory, twisted pair, Reed-Solomon error-correcting coding, PAM, Python.